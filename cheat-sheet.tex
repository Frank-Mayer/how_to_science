\documentclass[11pt]{article}
\usepackage{hyperref}
\begin{document}
	\title{Cheat Sheet der Wissenschaft}
	\author{Frank Mayer, Vanessa Vieth}
	\date{\today}
	\maketitle
	\tableofcontents

	\pagebreak

	\section{Einleitung}
	Dieses Dokument ist eine Sammlung von Tipps und Tricks für das wissenschaftliche Arbeiten.
	Bei Fragen zu \LaTeX{} ist die Overleaf Dokumentation oder die \LaTeX{} Projektseite hilfreich.
	\begin{itemize}
		\item \href{https://de.overleaf.com/learn}{https://de.overleaf.com/learn}
		\item \href{https://www.latex-project.org/help/documentation/}{https://www.latex-project.org/help/documentation}
	\end{itemize}

	\section{Wo bekomme ich wissenschaftliche Literatur her?}
	\subsection{Google Scholer}
	Google Scholer ist eine kostenlose Suchmaschine für wissenschaftliche Literatur. Sie durchsucht wissenschaftliche Zeitschriften, Bücher und Konferenzbeiträge. Die Suchergebnisse sind oft besser als bei der normalen Google-Suche, da Google Scholer nur wissenschaftliche Literatur durchsucht. \cite{halevi2017suitability}

	\href{htts://scholar.google.de/}{https://scholar.google.de/}

	\subsection{Semantic Scholar}
	https://www.semanticscholar.org/

	\pagebreak

	\bibliography{mybib}{}
	\bibliographystyle{plain}
\end{document}
